\documentclass[english,11pt,twoside,a4paper]{article}
\usepackage[left=2cm,top=1cm,right=2cm,nohead,nofoot]{geometry}
\usepackage[utf8]{inputenc}
\usepackage{hyperref}
\usepackage{amssymb}
\usepackage{graphicx}
\begin{document}
\author{
  Niemistö, Jesse
  \and
  Muona, Leo
  \and
  Hilden, Matias
}
\title{Evaluation}

\maketitle

\section{Dropped features}

Due to both algorithmic and mechanical difficulties, we had to drop the vertical rotation motor.

\section{Testing}

We performed two kind of tests on the project: Use case tests for determing wether the prototype works as intended, and perfomance testing to measure cycle speed during rest (no motion) and calculation (motion).

\subsection{Use case testing}

Use case testing is done in a room where at the start of the test the system has been initialized and program ran. Motor rotation at the initialization should be at the center of robot. Use case tests are identified with UC-prefix and index number suffix.

\textbf{UC-1: No motion test}

The first use case test is a test where no movement is at camera's field of view. The correct result is camera taking pictures (indicated via red blinking light) and no movement by the motor or sound from the audio system is played. The test result was successful. 

\textbf{UC-2: Left-side motion test}

The second use case test is a test where there is a single movement at left side of camera's field of view. Correct result contains the following characteristics:

\begin{itemize}
  \item Camera: Takes two pictures from which the movement is recognized. Then halts the picture taking for the time of motor rotation.
  \item Sound: A random sound clip is played to greet the person who has done the movement.
  \item Motor: Motor axis rotates the camera to left so that the center of camera points at movement spot.
\end{itemize}

The test result was "okay". That is, the correct motor rotation was made and soundclip was played. However targetting is not very accurate.

\textbf{UC-3: Right-side motion test}

The third use case test is a test where there is a single movement at right side of camera's field of view. Correct result contains the following characteristics:

\begin{itemize}
  \item Camera: Takes two pictures from which the movement is recognized. Then halts the picture taking for the time of motor rotation.
  \item Sound: A random sound clip is played to greet the person who has done the movement.
  \item Motor: Motor axis rotates the camera to right so that center of camera points at movement spot.
\end{itemize}

The test result was "okay". Rotation was made towards the correct spot and soundclip was played. However, like in previous test, targetting is not very accurate.

\textbf{UC-4: Maximum rotation test}

The fourth use case test is a test to check maximum rotation edges. This test includes movement done first on the left side of the camera, just in camera's field of view, multiple times untill camera no longer rotates to left. Then, same is done at the right side of the camera's field of view untill camera has turned multiple times untill camera no longer rotates to right. Test case correct result includes the following:

\begin{itemize}
  \item Camera: Takes pictures every time movement is recognized.
  \item Sound: Plays audio clip every time movement is recognized.
  \item Motor: Camera turns left multiple times when movement is recognized, then stops at 30 degrees from the center, which is the safe limit. When movement is done at the right side of camera, camera turns right multiple times when movement is recognized, then stops at 30 degrees from the center, which is the safe limit.
\end{itemize}

Test result was successful. As a sidenote, when running the test case again, the camera moves different lengths at a time.

\subsection{Performance testing}

Perfomance tests will programmically time the length picture cycles. Each cycle contains the following: taking a new picture, calculating differences between old and new pictures and calculating the rotation angle in case motion was detected. Rotation and audio playback are left out of the cycle as they depend on rotation and audio file length, thus impacting the results in a non-informative way.

\textbf{No motion}

The turret was set to watch a direction with no motion and the program was started. The results are as follows:

\textbf{Motion}

The turret was set to watch one of our project members who created motion. The results are as follows:

\section{Improvements}

Improvements to the project can be made on software side of things. The following issues can be improved:

\begin{itemize}
  \item Slow camera usage: This issue can be resolved by implementing the used raspistill program in our own code. This can be done relatively easily as raspistill is an open source program and does not conflict with our software licensing (GPLv3). It merely requires time.
  \item Inaccurate motion detection algorithm: This issue is a bit harder to solve, but it can propably be done by precomputing static scenery. This way we should be able to know where the moving object is instead of the algorithm showing two moving objects (starting and end location of the moving object).
  \item Low audio quality: Resolving this issue requires a firmware fix to the Raspberry Pi. Should such fix be made available, we can easily update it to our project.
\end{itemize}

\end{document}
